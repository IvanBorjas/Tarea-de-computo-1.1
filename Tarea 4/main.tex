\documentclass[4apaper, 12pt]{article}
\usepackage[utf8]{inputenc}

\title{Guía de estudio}
\author{Almazan Borjas Jorge Ivan }
\date{October 2022}
\usepackage[spanish]{babel}
\usepackage[dvipsnames]{xcolor}
\usepackage{amssymb}
\usepackage{mathrsfs}
\usepackage{mathptmx}
\usepackage{mathtools}
\usepackage{fancyhdr}
\pagestyle{fancy}
\fancyhf{}
\lfoot{\thepage}
\rhead{Ivan Borjas}
\lhead{\leftmark}


\begin{document}
\maketitle

\section{Física}
  
  \begin{itemize}
      \item {Fuerza de gravedad}
   \end{itemize}
   \begin{equation}
       \vec{F}=-G\dst\dfrac{m_1m_2}
       {|r|^2}\vec{e_r}\\
   \end{equation}
   F es la fuerza de la gravedad, M es la masa de un objeto, m es la masa de otro objeto, r es la distancia entre ellos, 
   $y G = 6.672 x 10-11Nm2/kg2$
   es una constante llamada constante de gravitación universal de Newton. 
   Esta relación rige el movimiento de los planetas en sus órbitas, guía las naves espaciales a sus destinos, e incluso mantiene nuestros pies en el suelo. 
   \begin{itemize}
            \item{ley de hooke} 
   \end{itemize}
      \begin{equation}
         \vec{F}=-k\vec{\Delta x}
      \end{equation} 
    La Ley de Hooke es sumamente importante en diversos campos, como en la física y el estudio de resortes elásticos (su demostración más frecuente).
    Permite prever la manera en que una fuerza prolongada o un peso alterará las dimensiones de los objetos en el tiempo. 
   \begin{itemize}
       \item{Momento angular} 
   \end{itemize}   
    \begin{equation}
        \vec{L}=\vec{r}\times m\vec{v}=
        \vec {r}\times\vec{p}=I\vec{\omega}
    \end{equation}
    El momento angular es una magnitud vectorial que utilizamos en física para caracterizar el estado de rotación de los cuerpos. En este apartado trataremos los siguientes puntos
   \begin{itemize}
       \item{energia cinetica} 
   \end{itemize}
    \begin{equation}
        E_c=\frac12mv^2
    \end{equation}
    Se define como la cantidad de trabajo realizado por todas las fuerzas que actúan sobre un cuerpo con una masa determinada, necesario para acelerarlo desde una velocidad inicial hasta otra velocidad final. Una vez alcanzada dicha velocidad, según la Ley de la inercia, la cantidad de energía cinética acumulada permanecerá constante, es decir, no variará, a menos que otra fuerza nuevamente actúe sobre el cuerpo, ejerciendo un trabajo sobre él, cambiando su velocidad y, por lo tanto, su energía cinética.
    La energía cinética a menudo se representa con el símbolo $E_c$ (puede ser $E_+$ o $E_{–}$, dependiendo del caso), aunque a veces también se utilicen los símbolos T o K. Suele expresarse en Joules (J).
   \begin{itemize}
       \item{Impulso}
   \end{itemize}
   \begin{equation}
       \vec I=\vec F\Delta t= m \Delta
       \vec v=\Delta \vec p
   \end{equation}
   El impulso es un término que cuantifica el efecto general de una fuerza que actúa con el tiempo. De manera convencional se le da el símbolo \text{J} y se expresa en newton-segundos.
   \begin{itemize}
       \item{Conservacion del momento angular}
   \end{itemize}
      \begin{equation}
          L=m_1 \cdot r_1 \cdot v_1=m_2 \cdot r_2 \cdot v_2
      \end{equation}
    El momento angular inicial total es igual al momento angular final total para un sistema sin torca externa neta. Comúnmente llamado la conservación del momento angular.

    \begin{itemize}
        \item{Tensión en n poleas moviles}
    \end{itemize}
    \begin{equation}
        F_i=\dfrac P{2^n}
    \end{equation}
    Como su nombre lo indica calcula la tension en poleas moviles que no es otra cosa que una polea de gancho conectada a una cuerda que tiene uno de sus extremos anclado a un punto fijo y el otro (extremo movil) conectado a un mecanismo de tracción.

    Estas poleas disponen de un sistema armadura-eje que les permite permanecer unidas a la carga y arrastrarla en su movimiento (al tirar de la cuerda la polea se mueve arrastrando la carga).
    \begin{itemize}
        \item {rendmiento energetico}
    \end{itemize}
    \begin{equation}
        \textcolor{purple}{\eta=\dfrac{E_{util}}{E_{tot}}=\dfrac{E_{tot}-E_{Perd}}{E_{total}}=\dfrac{P_{util}}{P_{total}}
        =\dfrac{P_{tot}-P_{Perd}}{P_{tol}}}
    \end{equation}
    En toda transformación parte de la energía se convierte en calor o energía térmica.
    Cualquier tipo de energía puede transformarse íntegramente en calor; pero, éste no puede transformarse íntegramente en otro tipo de energía. Se dice, entonces, que el calor es una forma degradada de energía.
    Se define, el Rendimiento como la relación (en \% por ciento) entre la energía útil obtenida y la energía aportada en una transformación.
    \begin{itemize}
        \item {Periodo del pendulo}
    \end{itemize}
    \begin{equation}
        {T=2\pi\sqrt{\dfrac{L}{g}}}
    \end{equation}
    El perido del pendulo se expresa como en funcion de la longitud del hilo ${l}$ y de la aceleración de la gravedad ${g}$
    \begin{itemize}
        \item {Segunda ley de Newton}
    \end{itemize}
    \begin{equation}
    \vec{F}=m\vec{a}= m \ddot{\vec x}
    \end{equation}
    “La aceleración de un objeto es directamente proporcional a la fuerza que actúa sobre él e inversamente proporcional a la masa”.

    Eso significa que para que un objeto se mueva rápidamente debes aplicarle mucha fuerza, pero también,  que la rapidez con la que se mueve el objeto depende de qué tan liviano o pesado es. 
    \begin{itemize}
        \item {Tercera ley de Newton}
    \end{itemize}
    \begin{equation}
        \vec {F}=-\vec {F}
    \end{equation}
    Establece que cuando dos partículas interactúan, la fuerza sobre una partícula es igual y opuesta a la fuerza que interactúa sobre la otra partícula. Es decir, si existe una fuerza externa, tal fuerza será contrarrestada por otra igual, pero en la dirección opuesta. Osea que la fuerza uno es igual a la 2 pero en direcciones opuestas
    \begin{itemize}
        \item [\clubsuit] {Peso}
    \end{itemize}
    \begin{equation}
        \vec{F}=m\vec{g}
    \end{equation}
    Aquí no hay mucho que decir se calcula multiplicando la masa (m) por el valor aproximado de la fuerza de gravedad (g) que varía de unos lugares a otros. Peso (P) = masa (m) x fuerza de gravedad (g).
    \begin{itemize}
        \item [\checkmark] {Energía potencial gravitatoria}
    \end{itemize}
    \begin{equation}
        E_p=U=mgh
    \end{equation}
    La energía potencial gravitatoria es la energía potencial asociada con el campo gravitatorio. Esta dependerá de la altura relativa de un objeto a algún punto de referencia, la masa y la aceleración de la gravedad
    \begin{itemize}
        \item[\heartsuit]{Periodo oscilado armonico}
    \end{itemize}
    \begin{equation}
        T=2\pi\sqrt{\dfrac{m}{k}}
    \end{equation}
    Es un sistema que, cuando se desplaza de su posición de equilibrio, experimenta una fuerza restauradora F proporcional al desplazamiento x: donde k es una constante positiva.
    \begin{itemize}
        \item [\clubsuit] {Energía mecanica de un satelite}
    \end{itemize}
    \begin{equation}
      E=\frac{1}{2}m_sv^{2}-G\frac{Mm_s}{r}
    \end{equation}
    Energía mecánica de un satélite ms  que orbita alrededor de un planeta de masa M, a una distancia r del mismo.
    \begin{itemize}
        \item[\bigstar]{Velocidad de escape}
    \end{itemize}   
    \begin{equation}
        V_e=\sqrt{2g_\omicron R}
    \end{equation}
    \begin{itemize}
        \item[\checkmark]{Tercera ley de Kepler}
    \end{itemize}
    La velocidad de escape es la velocidad inicial que hay que imprimirle a un objeto cualquiera para alejarse indefinidamente de un cuerpo o sistema más masivo al cual le vincula únicamente la gravedad.
    \begin{equation}
        T^{2}=\frac{4\pi^{2}}{GM}r^{3}
    \end{equation}
    \begin{itemize}
        \item {volumen de una esfera}
    \end{itemize}
    \begin{equation}
        \frac{4}{3}\pi R^{3}
    \end{equation} 
    Se usa para calcular el volumen de una esfera de radio R
    \begin{itemize}
        \item[\heartsuit] {Norma de un vector}
    \end{itemize}
    \begin{equation}
     ||\vec{v}|| = \sqrt{A^2\vec{i} + B^2\vec{j} + C^2\vec{k}}
    \end{equation}
    La definición general de norma se basa en generalizar a espacios vectoriales abstractos la noción de módulo de un vector de un espacio euclídeo. Se usa para obtener la magnitud de un vector obteninedo como resultado un escalar
 \section{Geometría}
    \begin{itemize}
        \item[\blacklozenge]{La chicharronera}
    \end{itemize}
    \begin{equation}
        \textcolor{Red} {x = \frac {-b \pm \sqrt {b^2 - 4ac}}{2a}}
    \end{equation}
    Es la formula general apra encontrar dos soluciones a una ecuación cuadratica ($a x^2 + b x + c = 0$), desconozco porque le dicen chicharronera jajaja
    \begin{itemize}
        \item {Pendiente de una recta}
    \end{itemize}
    \begin{equation}
        m=\frac{\triangle y}{\triangle x}=\frac{y_2-y_1}{x_2-x_1}	
    \end{equation}
    Es la ecuación que se usa para hayar la inclinacion de una recta en un plano 
    \begin{itemize}
        \item[\clubsuit]{Distancia entre puntos}
    \end{itemize}
    \begin{equation}
        \sqrt{(x_1 - x_2)^2 + (y_1 - y_2)^2}
    \end{equation}
    AL igual que la anterior se usa en un plano para calcular la distancia entre dos puntos del plano
    \begin{itemize}
        \item [\bigstar] {Teorema de Pitagoras}
    \end{itemize}
    \begin{equation}
      a^2+b^2=c^2
    \end{equation}
    El teorema de Pitágoras es una relación fundamental en geometría euclidiana entre los tres lados de un triángulo rectángulo. Es fundamental porque con el puedes resolver incognitas de diversos tipos de triangulos aunque no sean rectangulos
    \begin{itemize}
        \item [\checkmark]{Razones trigonometricas}
    \end{itemize}
    \begin{equation}
        \sin \alpha=\dfrac {\text{
        Cateto opuesto}}
        {\text{Hipotenusa}}=
        \sqrt{1-\cos^2\alpha}
        =\dfrac {\tan \alpha}
        {\sqrt{1+\tan^2\alpha}}
        \end{equation}
        \begin{equation*}
        \cos \alpha=\dfrac{\text{Cateto
        adyacente o contiguo}}
        {\text{Hipotenusa}}=
        \sqrt{1-\sin^2\alpha}=\dfrac {1}
        {\sqrt{1+\tan^2\alpha}}
        \end{equation*}
        \begin{equation*}
        \tan \alpha=\dfrac {\sin \alpha}
        {\cos \alpha}=\dfrac
        {\text{Cateto opuesto}}
        {\text{Cateto adyacente o
        contiguo}}
        \end{equation*}
        La noción de razón trigonométrica se refiere a los vínculos que pueden establecerse entre los lados de un triángulo que dispone de un ángulo de 90º. Existen tres grandes razones trigonométricas: tangente, seno y coseno.
        \begin{itemize}
            \item [\heartsuit]{Area de un hexagono}
        \end{itemize}
        \begin{equation}
            A= \frac{P\times{2}}{2}
        \end{equation}
        Se usa para calcular la formula de un hexagono 
\end{document}
